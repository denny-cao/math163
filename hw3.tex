%++++++++++++++++++++++++++++++++++++++++
\documentclass[article,12pt]{article}
\usepackage{float}
\usepackage{setspace}
\usepackage{tabu} % extra features for tabular environment
\usepackage{amsmath}  % improve math presentation
\usepackage{graphicx} % takes care of graphic including machinery
\usepackage[margin=1in]{geometry} % decreases margins
\usepackage{cite} % takes care of citations
\usepackage[final]{hyperref} % adds hyper links inside the generated pdf file
\usepackage{tikz}
\usepackage{caption} 
\usepackage{fancyhdr}
\usepackage{amssymb} % symbols like /therefore
\usepackage{amsthm} % proofs
\usepackage{enumerate} % lettered lists
\usepackage{mathtools} % macros
\usetikzlibrary{scopes}
\usepackage{xcolor} \pagecolor[rgb]{0.12549019607,0.1294117647,0.13725490196} \color[rgb]{0.82352941176,0.76862745098,0.62745098039}
\hypersetup{
	colorlinks=true,       % false: boxed links; true: colored links
	linkcolor=blue,        % color of internal links
	citecolor=blue,        % color of links to bibliography
	filecolor=magenta,     % color of file links
	urlcolor=blue         
}

\newcounter{question}
\setcounter{question}{0}
\def\questionnum{{\Large\bfseries{Question \arabic{question} }}\\[1em]}

\newcommand{\question}{
    \stepcounter{question}
    \questionnum
}

\newcommand{\solution}{
    {{\large\bfseries{Solution}}}
}

\newcommand{\hwTitle}{Practice Exam 3}
\newcommand{\hwDueDate}{November 14, 2022}
\newcommand{\class}{MATH163}
\newcommand{\professor}{Dr.\ Petrescu}
\newcommand{\name}{Denny\ Cao}

\pagestyle{fancy}
\fancyhf{}% clears all header and footer fields
\fancyfoot[C]{--~\thepage~--}
\renewcommand*{\headrulewidth}{0.4pt}
\renewcommand*{\footrulewidth}{0pt}
\lhead{\name}
\chead{\class: \hwTitle}
\rhead{Question \arabic{question}}


\fancypagestyle{plain}{%
  \fancyhf{}% clears all header and footer fields
  \fancyfoot[C]{--~\thepage~--}%
  \renewcommand*{\headrulewidth}{0pt}%
  \renewcommand*{\footrulewidth}{0pt}%
}


% Shortcuts
\DeclarePairedDelimiter\ceil{\lceil}{\rceil} % ceil function
\DeclarePairedDelimiter\floor{\lfloor}{\rfloor} % floor function

\DeclarePairedDelimiter\paren{(}{)} % parenthesis

\newcommand{\df}{\displaystyle\frac} % displaystyle fraction
\newcommand{\qeq}{\overset{?}{=}} % questionable equality

\newcommand{\Mod}[1]{\;\mathrm{mod}\; #1} % modulo operator

% Sets
\DeclarePairedDelimiter\set{\{}{\}}
\newcommand{\unite}{\cup}
\newcommand{\inter}{\cap}

\newcommand{\reals}{\mathbb{R}} % real numbers: textbook is Z^+ and 0
\newcommand{\ints}{\mathbb{Z}}
\newcommand{\nats}{\mathbb{N}}
\newcommand{\rats}{\mathbb{Q}}


% Counting
\newcommand\perm[2][^n]{\prescript{#1\mkern-2.5mu}{}P_{#2}}
\newcommand\comb[2][^n]{\prescript{#1\mkern-0.5mu}{}C_{#2}}

\setlength\parindent{0pt}

% Sign Charts
\newdimen\tcolw \tcolw=2.5em % the column width
\edef\ecatcode{\catcode`&=\the\catcode`&\relax}\catcode`&=4
\def\sgchart#1#2{\vbox{\offinterlineskip\halign{\hfil##\quad&##\hfil\crcr\sgchartA#2,:,%
   \omit\sgchartR&\kern.2pt\sgchartS{.5\tcolw}\relax\sgchartE#1,\relax,%
   \sgchartS{.5\tcolw}\relax\cr
   \noalign{\kern2pt}&\def~{}\kern.5\tcolw\sgchartD#1,\relax,\cr}}}
\def\sgchartA#1:#2,{\cr\ifx,#1,\else $#1$&\sgchartB#2{}\expandafter\sgchartA\fi}
\def\sgchartB#1{\hbox to\tcolw{\hss$#1$\hss}\sgchartC}
\def\sgchartC#1{\ifx,#1,\else
   \strut\vrule\kern-.4pt\hbox to\tcolw{\hss$#1$\hss}\expandafter\sgchartC\fi}
\def\sgchartD#1#2,{\ifx\relax#1\else\hbox to\tcolw{\hss$#1#2$\hss}\expandafter\sgchartD\fi}
\def\sgchartE#1#2,{\ifx\relax#1\else
    \ifx~#1\sgchartS\tcolw\circ \else\sgchartS\tcolw\bullet\fi \expandafter\sgchartE\fi}
\def\sgchartR{\leaders\vrule height2.8pt depth-2.4pt\hfil}
\def\sgchartS#1#2{\hbox to#1{\kern-.2pt\sgchartR \ifx\relax#2\else
   \kern-.7pt$#2$\kern-.7pt\sgchartR\fi\kern-.2pt}}
\ecatcode
%++++++++++++++++++++++++++++++++++++++++

\title{\hwTitle}
\author{\name}
\date{%
    %CCP MATH163 \\%
    %Dr. Petrescu \\[2ex]%
    Due: \hwDueDate
}
\begin{document}
\maketitle
% Question 1
\question 
Use the Principle of Mathematical Induction to show this inequality is true for all  integers $n\geq 2$:\hskip1cm$\displaystyle\sum_{i=1}^n\df{1}{\sqrt{i\,\,}} > \sqrt{n}$ \\
\solution
\begin{proof}
    By induction. \\
    Let P(n) be the statement that $\displaystyle\sum_{i=1}^n\df{1}{\sqrt{i\,\,}} > \sqrt{n\,\,}$, where $n \in \ints$. \\ 
    \textbf{Basis Step: $n=2$}
    \begin{align*}
        \sum_{i=1}^{2}{\frac{1}{\sqrt{i}}}&\stackrel{?}{>} \sqrt{2} \\
        \frac{1}{\sqrt{1}} + \frac{1}{\sqrt{2}}&\stackrel{?}{>}\sqrt{2} \\
        1 + \frac{\sqrt{2}}{2}&\stackrel{?}{>}\sqrt{2} \\
        2 + \sqrt{2}&\stackrel{?}{>}2\sqrt{2} \\
        2 &> \sqrt{2}
    \end{align*}
    Since $2 > \sqrt{2}$, $P(2)$ is true. \\
    \\
    \textbf{Inductive Hypothesis:} Assume $P(k)$ is true, $k \in \ints \land k \geq 2$. \\
    \\
    \textbf{Inductive Step:} We want to show that $P(k) \implies P(k+1)$. Assume the Inductive Hypothesis. \\
    \begin{align*}
        \sum_{i=1}^{k}{\frac{1}{\sqrt{i}}} &> \sqrt{k} \\
        1 + \cdots + \frac{1}{\sqrt{k}} &> \sqrt{k}
        \intertext{Let $c = 1 + \cdots + \df{1}{\sqrt{k}}$}
        c &> \sqrt{k} \\
        c + \frac{1}{\sqrt{k+1}} &> \sqrt{k} + \frac{1}{\sqrt{k+1}} \\
        c + \frac{1}{\sqrt{k+1}} &> \frac{\sqrt{k}\sqrt{k+1} + 1}{\sqrt{k+1}} \\
        c + \frac{1}{\sqrt{k+1}} &> \frac{\sqrt{k^2 + 2k} + 1}{\sqrt{k+1}} \\
        \\
        \sum_{i=1}^{k+1}{\frac{1}{\sqrt{i}}} &\stackrel{?}{>} \sqrt{k+1} \\
        1 + \cdots + \frac{1}{\sqrt{k}} + \frac{1}{k + 1} &\stackrel{?}{>} \sqrt{k+1} \\
        c + \frac{1}{k + 1} &\stackrel{?}{>} \sqrt{k+1} \\
        c + \frac{1}{k + 1} &\stackrel{?}{>} \frac{\sqrt{k+1}\sqrt{k+1}}{\sqrt{k+1}} \\
        c + \frac{1}{k + 1} &\stackrel{?}{>} \frac{\sqrt{k^2 + 2k + 1}}{\sqrt{k+1}} \\
        \intertext{Since $c + \df{1}{\sqrt{k+1}} > \df{\sqrt{k^2 + 2k} + 1}{\sqrt{k+1}}$, by the transitive property, $\df{\sqrt{k^2 + 2k} + 1}{\sqrt{k+1}} > \df{\sqrt{k^2 + 2k + 1}}{\sqrt{k+1}} \implies c + \df{1}{\sqrt{k+1}} > \df{\sqrt{k^2 + 2k + 1}}{\sqrt{k+1}}$}
        \frac{\sqrt{k^2 + 2k} + 1}{\sqrt{k+1}} &\stackrel{?}{>} \frac{\sqrt{k^2 + 2k + 1}}{\sqrt{k+1}} \\
        \sqrt{k^2 + 2k} + 1 &\stackrel{?}{>} \sqrt{k^2 + 2k + 1} \\
        \paren*{\sqrt{k^2 + 2k} + 1}^2 &\stackrel{?}{>} k^2 + 2k + 1 \\
        k^2 + 2k + 1 + 2\sqrt{k^2 + 2k} &\stackrel{?}{>} k^2 + 2k + 1 \\
        \intertext{Since $k \geq 2$, the statement is true:}
        \frac{\sqrt{k^2 + 2k} + 1}{\sqrt{k+1}} &> \frac{\sqrt{k^2 + 2k + 1}}{\sqrt{k+1}}
        \intertext{Therefore, $c + \df{1}{\sqrt{k+1}} > \df{\sqrt{k^2 + 2k + 1}}{\sqrt{k+1}} = c + \df{1}{\sqrt{k+1}} > \sqrt{k+1} = \sum_{i=1}^{k+1}{\frac{1}{\sqrt{i}}} > \sqrt{k+1}$. Thus, $P(k) \implies P(k+1)$.}
    \end{align*}
    \textbf{Conclusion:} By induction, we have shown that $\displaystyle\sum_{i=1}^{n+1}{\frac{1}{\sqrt{i}}} > \sqrt{n+1}$ is true $n\geq2$
\end{proof}
\newpage
% Question 2
\question
Prove that for all positive integers n,  $3^{2^n} -1$ is divisible by $2^{n+2}$.
\begin{proof}
    By induction. \\
    Let $P(n)$ be the statement that $3^{2^n} -1$ is divisible by $2^{n+2}$, $n \in \ints^+$ \\ 
    \\
    \textbf{Basis Step: $n=1$}
    \begin{align*}
        (3^{2^1} - 1)\Mod{2^{1+2}} &= (3^{2} - 1)\Mod{2^{3}} \\
                                   &= 8\Mod{8} \\
                                   &= 0
    \end{align*}
    Since the remainder when $3^{2^1}-1$ is divided by $2^{1+2}$ is 0, $P(n)$ is true when $n=0$. \\
    \\
    \textbf{Inductive Hypothesis:} Assume $P(k)$ is true for some $k \in \ints^+$. \\
    \\
    \textbf{Inductive Step:} We want to show that $P(k) \implies P(k+1)$. Since we assume $P(k)$ is true in the Inductive Hypothesis, we must prove that $P(k+1)$ is true.\\
    \begin{align*}
        \frac{3^{2^{k+1}} - 1}{2^{k+1+2}} &= \frac{3^{2^k(2)} - 1}{2^1\paren*{2^{k+2}}} \\
                                          &= \frac{3^{2^{k^2}} - 1}{2^1\paren*{2^{k+2}}} \\
                                          &= \frac{\paren*{3^{2^k} - 1}\paren*{3^{2^k} + 1}}{2^1\paren*{2^{k+2}}} \\
                                          &= \paren*{\frac{3^{2^k} - 1}{2^{k+2}}}\paren*{\frac{3^{2^k} + 1}{2}}
    \end{align*}
    From the Inductive Hypothesis, $P(k)$ is true. Therefore, $\paren*{3^{2^k} - 1}\Mod{2^{k+2}} = 0$, meaning $\df{3^{2^k} - 1}{2^{k+2}} \in \ints$. Since $\paren*{\df{3^{2^k} - 1}{2^{k+2}}}\paren*{\df{3^{2^k} + 1}{2}}$ are factors of $\df{3^{2^{k+1}} - 1}{2^{k+1+2}}$, if both are integers, $\paren*{3^{2^{k + 1}} - 1}\Mod{2^{k+3}} = 0$, meaning $P(k + 1)$ is true. Therefore, it is sufficient to prove $P(k + 1)$ by proving $\paren*{3^{2^k} + 1}\Mod{2} = 0$:
    \begin{align*}
        \intertext{For an integer $x$ to be even, 2 must be included in the prime factorization of $x$. The prime factorization of $3^{2^k}$ is $2^k$ amount of 3s. Since 2 is not included and $3^{2^k} \neq 0$, $3^{2^k}$ is odd. Therefore, $3^{2^k}$ can be expressed as $2c + 1, c \in \ints$:}
        {[(2c + 1) + 1]}\Mod{2}  &= {(2c + 2)}\Mod{2} \\
                                 &= ((2c\Mod{2}) + 2\Mod{2})\Mod{2} \\
                                 &= (((2\Mod{2})(c\Mod{2}))\Mod{2} + 2\Mod{2})\Mod{2} \\
                                 &= ((0(c\Mod{2}))\Mod{2} + 0)\Mod{2} \\
                                 &= 0\Mod{2} \\
                                 &= 0
        \intertext{$((2c+1) + 1)\Mod{2} = 0$, where $2c+1 = 3^{2^k}$, therefore $\df{3^{2^k} + 1}{2} \in \ints$. Since both factors of $\df{3^{2^{k+1}} - 1}{2^{k+1+2}}$ are integers, $\df{3^{2^{k+1}} - 1}{2^{k+1+2}}$ will also be an integer. Therefore, $\paren*{3^{2^{k+1}} - 1}\Mod{2^{k+1+2}} = 0$; $P(k+1)$ is true. Thus, $P(k) \implies P(k+1)$.}
    \end{align*}

\end{proof}
\newpage
% Question 3
\question
Find a formula for $$(1-\df1{2^2})(1-\df1{3^2})(1-\df1{4^2})(1-\df1{5^2})...(1-\df1{n^2})$$  where $n \geq 2$, and use the Principle of Mathematical Induction to prove that the formula is correct. \\
\begin{align*}
      &\paren*{1-\df1{2^2}}\paren*{1-\df1{3^2}}\paren*{1-\df1{4^2}}\paren*{1-\df1{5^2}}\cdots \paren*{1-\df1{n^2}} \\
    = &\paren*{\frac{2^2-1}{2^2}}\paren*{\frac{3^2-1}{3^2}}\paren*{\frac{4^2-1}{4^2}}\paren*{\frac{5^2-1}{5^2}}\cdots \paren*{\frac{n^2-1}{n^2}} \\
    = &\paren*{\frac{(2+1)(2-1)}{2^2}}\paren*{\frac{(3+1)(3-1)}{3^2}}\paren*{\frac{(4+1)(4-1)}{4^2}}\paren*{\frac{(5+1)(5-1)}{5^2}}\cdots\paren*{\frac{(n+1)(n-1)}{n^2}} \\
    = &\paren*{\frac{3 \times 1}{2^2}}\paren*{\frac{4 \times 2}{3^2}}\paren*{\frac{5 \times 3}{4^2}}\paren*{\frac{6 \times 4}{5^2}}\cdots\paren*{\frac{(n+1)(n-1)}{n^2}} \\
    = &\frac{\df{(n+1)!}{2} \times (n-1)!}{n! \times n!} = \frac{(n+1)! \times (n-1)!}{2 \times n! \times n!} = \frac{n+1}{2n}
\end{align*}
\begin{proof} By induction. \\
    \[ \paren*{1-\df1{2^2}}\paren*{1-\df1{3^2}}\paren*{1-\df1{4^2}}\paren*{1-\df1{5^2}}\cdots \paren*{1-\df1{n^2}} = \prod_{i=2}^{n}{1 - \frac{1}{i^2}} \]
    Let P(n) be the statement that:
    \[ \prod_{i=2}^{n}{1 - \frac{1}{i^2}} = \frac{n+1}{2n}\]
    where $n \geq 2, n \in \ints$. \\
    \\
    \textbf{Basis Step: $n=2$}
    \begin{align*}
        \begin{aligned}
            \prod_{i=2}^{2}{1-\frac{1}{i^2}} &= \paren*{1 - \frac{1}{2^2}} \\
                                         &= \paren*{1 - \frac{1}{4}} \\
                                         &= \frac{3}{4}
        \end{aligned}
        &&
        \begin{aligned}
            \frac{2+1}{2(2)} &= \frac{3}{4} \\
            \\
            \\
            \\
        \end{aligned}
    \end{align*}
    Since $\displaystyle\prod_{i=2}^{n}{1-\frac{1}{i^2}}$ and $\df{n+1}{2n}$ evaluate to the same value, $\df{3}{4}$, when $n=2$, $P(2)$ is true. \\
    \\
    \textbf{Inductive Hypothesis:} Assume $P(k)$ is true, $k \geq 2, k \in \ints$. \\
    \\
    \textbf{Inductive Step:} We want to show that $P(k) \implies P(k+1)$. Since we assume $P(k)$ is true in the Inductive Hypothesis, we must prove that $P(k+1)$ is true: 
    \begin{align*}
        \prod_{i=2}^{k}{1-\frac{1}{i^2}} &= \frac{k+1}{2k} \\
        \paren*{\prod_{i=2}^{k}{1-\frac{1}{i^2}}}\paren*{1-\frac{1}{(k+1)^2}} &= \paren*{\frac{k+1}{2k}}\paren*{1-\frac{1}{(k+1)^2}} \\
        \prod_{i=2}^{k+1}{1-\frac{1}{i^2}} &= \paren*{\frac{k+1}{2k}}\paren*{\frac{(k+1)^2 - 1}{(k+1)^2}} \\
                                           &= \frac{(k+1)^2 - 1}{2k(k+1)} \\
                                           &= \frac{k^2 + 2k + 1 - 1}{2k(k+1)} \\
                                           &= \frac{k^2 + 2k}{2k(k+1)} \\
                                           &= \frac{k(k+2)}{2k(k+1)} \\
                                           &= \frac{k+2}{2(k+1)} \\
                                           &= \frac{(k + 1) + 1}{2(k+1)} 
    \end{align*} 
    $\displaystyle\paren*{\prod_{i=2}^{k}{1-\frac{1}{i^2}}}\paren*{1-\frac{1}{(k+1)^2}}$, which is equivalent to $\displaystyle\prod_{i=2}^{k+1}{1-\frac{1}{i^2}}$, equals $\df{(k + 1) + 1}{2(k+1)}$. Thus, $P(k) \implies P(k+1)$. \\
    \\
    \textbf{Conclusion:} By induction, we have shown that $P(n)$ is true, or $\displaystyle\prod_{i=2}^{n}{1-\frac{1}{i^2}} = \df{n+1}{2n}$, $n \geq 2, n \in \ints$.
\end{proof}
\newpage
% Question 4
\question
Which amounts of money can be formed using just two-dollar bills and five-dollar bills? Prove your answer using strong induction. \\
\\
\solution
\begin{proof}
    By strong induction. \\
    An amount of money, $n$, that can be formed using just two-dollar bills and five-dollar bills can be represented by $n=2a+5b$, where $a$ is the amount of two-dollar bills, and $b$ is the amount of five-dollar bills. \\
    \\
    Let $P(n)$ be the statement that $n=2a+5b$, $a, b \in \nats$. \\
    \\
    \textbf{Basis Step: $n \in \set*{2, 4, 5}$}
    \begin{alignat*}{2}
        \begin{aligned}
            n = 4 \\
            2a + 5b &= 4 \\
            2(2) + 5(0) &= 4 \\
        \end{aligned} \quad &&
        \begin{aligned}
            n = 5 \\
            2a + 5b &= 5 \\
            2(0) + 5(1) &= 5 \\
        \end{aligned}
    \end{alignat*}
    Since $2a+5b=4,5$ when $a=2,b=0$; $a=0,b=1$ respectively, $P(4)$ and $P(5)$ are true. \\
    \\
    \textbf{Inductive Hypothesis:} Assume $P(j)$ is true, $4 \leq j \leq k$, where $j,k \in \ints$. \\
    \\
    \textbf{Inductive Step:} We want to show that $P(k)\implies P(k+1)$. Since we assume $P(k)$ is true in the Inductive Hypothesis, we must prove that $P(k+1)$ is true.
    \begin{align*}
        2a + 5b = k+1 = (k - 1) + 2 
    \end{align*}   
    By adding a 2-dollar bill to $k-1$, we can form $k+1$. From the Inductive Hypothesis, $P(k-1)$ is true, meaning $k-1$ can be formed with just two-dollar bills and five-dollar bills. Therefore, since $k+1$ can be formed by adding another two-dollar bill, $P(k+1)$ is true. \\
    \\
    \textbf{Conclusion:} By strong induction, we have shown that $P(n)$ is true for $n \geq 4, n \in \ints$, or $n$ dollars can be formed with just two-dollar bills and five-dollar bills, $n \geq 4, n \in \ints$. \\
    \\
    2 dollars can be formed with 1 two-dollar bill and 0 five-dollar bills. Therefore, $n$ dollars can be formed with just two-dollar bills and five-dollar bills, $n \in \set*{2} \unite \set*{k \in \ints \ | \ k \geq 4}$.
\end{proof}
\newpage
% Question 5
\question
A baker bakes six different kinds of muffins. If a box with 25 muffins is made with a random number  of each kind of muffin, in how many ways can a box of muffins be prepared. \\
\\
\solution \\
\\
When repetition of elements is allowed, there are $C(n+r-1,r)$ $r$-combinations from a set with $n$ elements. We have $n=6$ kinds muffins and $r=25$ total muffins: 
\begin{align*}
    \binom{6 - 1 + 25}{25} &= \binom{30}{25} \\
                           &= 142,506
\end{align*}
\textbf{There are 142,506 ways for a box of muffins to be prepared.}
\newpage
% Question 6
\question
Let P(n) be the statement that a postage of n cents can be formed using just 3- cent stamps and 5- cent stamps. Parts of this exercise outline a strong induction proof that P(n) is true for $n \geq 8$.\\ a) Show that the statements P(8), P(9), and P(10) are true, completing the basis step of the proof. \\ b) What is the inductive hypothesis of the proof?\\ c) What do you need to prove in the inductive step?\\ d) Complete the inductive step for $k \geq 10$. \\
\\
\solution \\
\\
A postage of n cents that can be formed using just 3-cent stamps and 5-cent stamps can be represented by $n=3a+5b$, where $a,b$ are the amount of 3-cent stamps and 5-cent stamps respectively. \\
\begin{enumerate}[a)]
    \item 
    \textbf{Basis Step:}
    \begin{alignat*}{3}
        \begin{aligned}
            n = 8 \\
            3a + 5b &= 8 \\
            3(1) + 5(1) &= 8 \\
        \end{aligned} \quad &&
        \begin{aligned}
            n = 9 \\
            3a + 5b &= 9 \\
            3(3) + 5(0) &= 9 \\
        \end{aligned} \quad &&
        \begin{aligned}
            n = 10 \\
            3a + 5b &= 10 \\
            3(0) + 5(2) &= 10 \\
        \end{aligned}
    \end{alignat*}
    Since a postage of 8-cents can be formed with 1 3-cent stamp and 1 5-cent stamp, a postage of 9-cents can be formed with 3 3-cent stamps and 0 5-cent stamps, and a postage of 10-cents can be formed with 0 3-cent stamps and 2 5-cent stamps, $P(8),P(9),P(10)$ are true.
    \item 
    \textbf{Inductive Hypothesis:} Assume $P(j)$ is true, $8 \leq j \leq k$ where $j,k \in \ints$.
    \item 
    In the Inductive Step, we must prove that $P(k) \implies P(k+1)$. Since we assume $P(k)$ is true from the Inductive Hypothesis, we must prove that $P(k+1)$ is true.
    \item 
    \textbf{Inductive Step:} we want to prove that $P(k) \implies P(k+1)$. Since we assume $P(k)$ is true from the Inductive Hypothesis, we must prove that $P(k+1)$ is true: 
    \begin{align*}
        k + 1 &= (k - 2) + 3
    \end{align*}
\end{enumerate}
\newpage
% Question 7
\question
Find the number of strings of length 10 of letters of the alphabet, with no repeated letters, that have vowels in the first two positions. {\bf NOTE:} y is a vowel. \\
\\
\solution \\
\\
The cardinality of the set of vowels, $\set*{\text{a},\text{e},\text{i},\text{o},\text{u},\text{y}}$, is 6. The number of 2 permutations of 6 without repetition is $6 \times 5$. This is the amount of unique 2 letter strings, where both letters are in the set of vowels and are not the same. 
    \begin{align}
        \underset{\text{-----}}{6}\times\underset{\text{{-----}}}{5}
    \end{align}
The cardinality of the set of letters in the alphabet is 26. Since there cannot be repetition and 2 letters have been used, we are left with 24 letters. The number of 8 permutations of 24 without repetition is $\df{24!}{(24-8)!} = \df{24!}{16!}$. This is the amount of unique 8 letter strings excluding 2 letters and each letter is different.
    \begin{align}
        \underset{\text{{-----}}}{24}\times\underset{\text{{-----}}}{23}\times\underset{\text{{-----}}}{22}\times\underset{\text{{-----}}}{21}\times\underset{\text{{-----}}}{20}\times\underset{\text{{-----}}}{19}\times\underset{\text{{-----}}}{18}\times\underset{\text{{-----}}}{17}
    \end{align}
We can use the product rule, multiplying (1) and (2) to find the amount of strings of length 10 of letters of the alphabet, with no repeated letters, that have vowels in the first two positions: there are $\df{6!}{4!} \times \df{24!}{16!}$ strings.
\begin{align*}
    \underbrace{\underset{\text{-----}}{6}\times\underset{\text{{-----}}}{5}}_{(1)}\times\underbrace{\underset{\text{{-----}}}{24}\times\underset{\text{{-----}}}{23}\times\underset{\text{{-----}}}{22}\times\underset{\text{{-----}}}{21}\times\underset{\text{{-----}}}{20}\times\underset{\text{{-----}}}{19}\times\underset{\text{{-----}}}{18}\times\underset{\text{{-----}}}{17}}_{(2)}
\end{align*}
\newpage
% Question 8
\question
Ten men and ten women are to be put in a row. Find the number of possible different  rows if no two of the same sex stand adjacent. \\
\\
\solution \\
\\
We have $10!$ ways to organize the men and $10!$ ways to organize the women. Alternating the genders is the same as placing the 10 women in between 10 men or placing 10 men in between 10 women. This can be visualized as follows:
\begin{center}
    MWMWMWMWMWMWMWMWMWMW
\end{center}
\begin{center}
    WMWMWMWMWMWMWMWMWMWM
\end{center}
By the product rule, there are $10! \times 10!$ ways to order 10 women in between 10 men. Likewise, there are $10! \times 10!$ ways to order 10 men in between 10 women. Therefore, the total amount of different rows if no two of the same sex stand adjacent is the sum of these two possibilities: $10! \times 10! + 10! \times 10! = 2 \times 10! \times 10!$
\newpage
% Question 9
\question
If positive integers are chosen at random, what is the minimum number you must have in order to guarantee that two of the chosen numbers are congruent modulo 6.   Prove your answer. \par  {\bf NOTE:} Two numbers are congruent modulo 6 if their difference is a multiple of 6.\\
\\
\solution
\begin{proof}
    By definition of congruence modulo, $ a \equiv b \pmod{c} \iff a \Mod{c} = b \Mod{c}$. Therefore, two integers, $a$ and $b$, are congruent modulo 6 if and only if $a \Mod{6} = b \Mod{6}$. \\
    
    The codomain of the function $f(x) = x \Mod 6$ is $\set*{k, 0 \leq k \leq 5, k \in \ints}$. The cardinality of the codomain of $f$ is 6. Therefore, there are only 6 possible outputs for $f$. By the pigeonhole principle, with 7 random integers in the domain, at least 2 must map to the same value in the codomain---at least 2 are congruent modulo 6---because there are only 6 values in the codomain. Thus, there must be a minimum of 7 random positive integers to guarentee that two numbers are congruent modulo 6.

     There are $n$ possible remainders when an integer is divided by $n$. Therefore, there are 6 possible remainders when an integer is diviided by 6; $\set*{0,1,2,3,4,5}$. When there are 7 random integers, by the pigeon hole principle, there must be at least two with the same remainder when divided by 6 since there are only 6 possible remainders
\\
\\
    The codomain of the function $f(x) = x \Mod 6$ is $\set*{k, 0 \leq k \leq 5, k \in \ints}$. Suppose that for each element $y$ in the codomain of $f$, we have a box that contains all elements $x$ of the domain of $f$ such that $f(x)=y$. \\
    \\
    \\
    Since thecardinality of the codomain of $f$ is 6, the number of boxes is 6. By the pigeonhole principle, the number of random integers must be at least 7 to have one box contain 2 values from the domain.
\end{proof}
\newpage
% Question 10
\question
Write the expansion of $\paren*{x^2 -\df1 x}^{12}$.
\begin{align*}
    \paren*{x^2 - \frac{1}{x}}^{12} &= \sum_{i=0}^{12}\binom{12}{i}{x^2}^{12-i}\paren*{-\frac{1}{x}}^{i} \\
    &= \binom{12}{0} \paren*{x^{2}}^{12} \paren*{\frac{1}{x}}^{0} + \binom{12}{1} \paren*{x^{2}}^{11} \paren*{\frac{1}{x}}^{1} + \binom{12}{2} \paren*{x^{2}}^{10} \paren*{\frac{1}{x}}^{2} + \binom{12}{3} \paren*{x^{2}}^{9} \paren*{\frac{1}{x}}^{3} \\ + \binom{12}{4} \paren*{x^{2}}^{8} \paren*{\frac{1}{x}}^{4} + \binom{12}{5} \paren*{x^{2}}^{7} \paren*{\frac{1}{x}}^{5} + \binom{12}{6} \paren*{x^{2}}^{6} \paren*{\frac{1}{x}}^{6} \\ + \binom{12}{7} \paren*{x^{2}}^{5} \paren*{\frac{1}{x}}^{7} + \binom{12}{8} \paren*{x^{2}}^{4} \paren*{\frac{1}{x}}^{8} + \binom{12}{9} \paren*{x^{2}}^{3} \paren*{\frac{1}{x}}^{9} \\+ \binom{12}{10} \paren*{x^{2}}^{2} \paren*{\frac{1}{x}}^{10} + \binom{12}{11} \paren*{x^{2}}^{1} \paren*{\frac{1}{x}}^{11} + \binom{12}{12} \paren*{x^{2}}^{0} \paren*{\frac{1}{x}}^{12} \\ 
    \\
    &= 1 \paren*{x^{2}}^{12} \paren*{\frac{1}{x}}^{0} + 12 \paren*{x^{2}}^{11} \paren*{\frac{1}{x}}^{1} + 66 \paren*{x^{2}}^{10} \paren*{\frac{1}{x}}^{2} + 220 \paren*{x^{2}}^{9} \paren*{\frac{1}{x}}^{3} + 495 \paren*{x^{2}}^{8} \paren*{\frac{1}{x}}^{4} + 792 \paren*{x^{2}}^{7} \paren*{\frac{1}{x}}^{5} + 924 \paren*{x^{2}}^{6} \paren*{\frac{1}{x}}^{6} + 792 \paren*{x^{2}}^{5} \paren*{\frac{1}{x}}^{7} + 495 \paren*{x^{2}}^{4} \paren*{\frac{1}{x}}^{8} + 220 \paren*{x^{2}}^{3} \paren*{\frac{1}{x}}^{9} + 66 \paren*{x^{2}}^{2} \paren*{\frac{1}{x}}^{10} + 12 \paren*{x^{2}}^{1} \paren*{\frac{1}{x}}^{11} + 1 \paren*{x^{2}}^{0} \paren*{\frac{1}{x}}^{12}
\end{align*}
    \newpage
% Question 11
\question
In how many ways can 7 of the 8 letters in CHEMISTS be put in a row? \\
\\
\solution
\\
The word CHEMISTS has 8 letters, not all of them distinct. To create a 7 letter word, we can consider 2 cases: \\
\\
\textbf{Case 1:} Selecting 2 S's \\
There are $\displaystyle\binom{7}{2}$ ways to choose 2 slots where the 2 S's will go. That leaves 5 positions for 6 letters, which there are $\perm[6]{5}= \df{6!}{(6-5)!} = 6!$ ways to arrange them. By the product rule, there are $\displaystyle\binom{7}{2} \times 6!$ ways to make a 7 letter word from the letters of CHEMISTS, given that we select 2 S's.\\
\\
\textbf{Case 2:} Selecting 1 S \\
There are $\displaystyle\binom{7}{1}$ ways to choose 1 slot where the S will go. That leaves 6 positions for 6 letters (There are 7 letters left, but since we are considering only 1 S, we remove the remaining S from the count as well.), which there are $\perm[6]{6} = \df{6!}{(6-6)!} = 6!$ ways to arrange them. By the product rule, there are $\displaystyle\binom{7}{1} \times 6!$ ways to make a 7 letter word from the letters of CHEMISTS, given that we select only 1 S. \\
\\
We can add these two cases together to get the total ways: 
\begin{align*}
    \binom{7}{2} \times 6! + \binom{7}{1} \times 6! &= \frac{7!\enspace 6!}{(7-2)! \enspace 2!} + \frac{7! \enspace 6!}{(7-1)! \enspace 1!} \\
                                                    &= 7! \times 3 + 7! \\
                                                    &= 7! \times 4 \\
                                                    &= 20160
\end{align*}
\textbf{There are 20160 ways to put 7 of the 8 letters in CHEMISTS in a row.}
\newpage
% Question 12
\question 
What is the minimum number of cards that must be drawn from an ordinary deck of cards to guarantee that you have been dealt \\ (a) at least three of at least one suit?\\ (b) at least three clubs?  \\ Explain. \\
\\
\solution
\begin{enumerate}[a)]
    \item 
    A minimum of 9 cards must be drawn to guarentee at least three of at least one suit. Suppose there is a box for each suit (There will be 4 boxes), and each time a card is drawn, it is placed in the box that corresponds to its suit. With 8 cards, it is possible to have 2 cards of each suit, therefore more than 8 cards are needed. By the generalized pigeonhole principle, with at least 9 objects, at least three objects will be in one box---at least three cards that are the same suit. We can verify this algebraically: \\
    \\
    By generalized pigeonhole principle, if $N$ objects are placed in $k$ boxes, then there is at least one box containing at least $\ceil*{N/k}$ objects. Let $k$ be the number of suits; $k=4$. \\
    \\
    \[ \ceil*{\frac{N}{4}} \geq 3 \]
    \begin{align*}
        N &\geq 2 \times 4 + 1 \\
        N &\geq 9
    \end{align*}
    Thus, a minimum of 9 cards must be drawn to guarantee at least three of at least one suit.
    \item 
    A minimum of 42 cards must be drawn to guarentee at least three clubs. Note that in the worst case, we can select all the hearts, diamonds, and spades, a total of 39 cards, before we select a single club. The next 3 cards will be all clubs.  
\end{enumerate}
\newpage
% Question 13
\question
Prove the identity $\binom{n}{r}\binom{r}{k}=\binom{n}{k}\binom{n-k}{r-k}$ , whenever $n, r$, and $k$ are nonnegative integers with $r \leq n$ and $k \leq r$,\\a) algebraically.\\ b) using a combinatorial argument. \\
\\
\solution 
\begin{enumerate}[a)]
    \item 
    \begin{proof}
        
        \begin{align*}
            \binom{n}{k}\binom{n-k}{r-k} &= \paren*{\frac{n!}{(n-k)! \enspace k!}}\paren*{\frac{(n-k)!}{(n-k - (r-k))! \enspace (r-k)!}} \\
                                         &= \paren*{\frac{n!}{(n-k)! \enspace k!}}\paren*{\frac{(n-k)!}{(n-r)! \enspace (r-k)!}} \\
                                         &= \frac{n!}{(n-r)! \enspace (r-k)! \enspace k!} \\
                                         &= \paren*{\frac{n!}{(n-r)! \enspace (r-k)! \enspace k!}}\paren*{\frac{r!}{r!}} \\
                                         &= \paren*{\frac{n!}{(n-r)! \enspace r!}}\paren*{\frac{r!}{(r-k)! \enspace k!}} \\
                                         &= \binom{n}{r}\binom{r}{k}
        \end{align*}
        Thus, we have proved the identity $\displaystyle\binom{n}{r}\binom{r}{k}=\binom{n}{k}\binom{n-k}{r-k},n,r,k \in \nats \land r \leq n \land k \leq r$ algebraically.
    \end{proof}
\end{enumerate}
\newpage
% Question 14
\question
How many solutions are there to the equation $\sum_{i=1}^6 x_i=29$, where $x_i, i = 1, 2, 3, 4, 5, 6,$ is a nonnegative integer such that\\ a)$x_i > 1$ for $i = 1, 2, 3, 4, 5, 6$?\\ b)$x_1 \geq 1$, x2 $\geq 2$, $x_3 \geq 3$, $x4 \geq 4$, $x_5 > 5$, and $x_6 \geq 6$?\\ c)$x_1 \geq 5$? \\ d)$x_1 < 8$ and $x_2 > 8$? \\
\\
\solution \\
\\
Let $S=\set*{1,2,3,4,5,6}$
\begin{enumerate}[a)]
    \item 
    Since $x_i > 1$, there are at least 2 integers distributed among $x_i, i \in S$. Thus, there is 23 left to be distributed amongst $x_i, i \in S$:
    \begin{align*}
        \binom{6-1+29-2(6)}{29-2(6)} = \binom{22}{17}
    \end{align*}
    \item 
    There are at least $1,2,3,4,6,6$ distributed to $x_i, i \in S$ respectively. Thus, there is 22 left to be distributed amongst $x_i, x_i, i \in S$:
    \begin{align*}
        \binom{6-1+29-(1+2+3+4+6+6)}{29-(1+2+3+4+6+6)} = \binom{12}{7}
    \end{align*}
    \item
    Since $x_1 \geq 5$, there are at least 5 integers distributed to $x_1$. Thus, there is 24 left to be distributed amongst $x_i, i \in S$: 
    \begin{align*}
        \binom{6-1+29-5}{29-5} = \binom{29}{24}
    \end{align*}
    \item 

\end{enumerate}
\end{document} 

