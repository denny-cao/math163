%++++++++++++++++++++++++++++++++++++++++
\documentclass[article, 11pt]{article}
\usepackage{float}
\usepackage{setspace}
\usepackage{tabu} % extra features for tabular environment
\usepackage{amsmath}  % improve math presentation
\usepackage{graphicx} % takes care of graphic including machinery
\usepackage[margin=1in]{geometry} % decreases margins
\usepackage{cite} % takes care of citations
\usepackage[final]{hyperref} % adds hyper links inside the generated pdf file
\usepackage{tikz}
\usepackage{caption} 
\usepackage{fancyhdr}
\usepackage{amssymb} % symbols like /therefore
\usepackage{amsthm} % proofs
\usepackage{enumerate} % lettered lists
\usepackage{mathtools} % macros
\usetikzlibrary{scopes}
\usepackage{xcolor} \pagecolor[rgb]{0.12549019607,0.1294117647,0.13725490196} \color[rgb]{0.82352941176,0.76862745098,0.62745098039} % dark theme
\theoremstyle{definition}
\newtheorem{theorem}{Theorem}[section]
\newtheorem{definition}{Definition}[section]
\hypersetup{
	colorlinks=false,      % false: boxed links; true: colored links
	linkcolor=blue,        % color of internal links
	citecolor=blue,        % color of links to bibliography
	filecolor=magenta,     % color of file links
	urlcolor=blue         
}
\usepackage{titling}
\renewcommand\maketitlehooka{\null\mbox{}\vfill}
\renewcommand\maketitlehookd{\vfill\null}
\usepackage{siunitx} % units
\usepackage{verbatim} 
\newcommand{\studyTitle}{Study Guide} 
\newcommand{\class}{MATH 163: Discrete Mathematics 1 Fall 2022}
\newcommand{\professor}{Dr. Petrescu}
\newcommand{\name}{Denny Cao}
\newcommand{\final}{Monday, December 12, 2022}
\pagestyle{fancy}
\fancyhf{}% clears all header and footer fields
\fancyfoot[C]{--~\thepage~--}
\renewcommand*{\headrulewidth}{0.4pt}
\renewcommand*{\footrulewidth}{0pt}
\lhead{\name}
\chead{\leftmark}
\rhead{\professor}


\fancypagestyle{plain}{%
  \fancyhf{}% clears all header and footer fields
  \fancyfoot[C]{--~\thepage~--}%
  \renewcommand*{\headrulewidth}{0pt}%
  \renewcommand*{\footrulewidth}{0pt}%
}

% Shortcuts
\newcommand{\xor}{\oplus} % exclusive or
\newcommand{\true}{\textbf{T}} % true
\newcommand{\false}{\textbf{F}} % false
\newcommand{\lra}{\leftrightarrow} % iff

\newcommand{\powset}{\mathcal{P}} % power set

\DeclarePairedDelimiter\ceil{\lceil}{\rceil} % ceil function
\DeclarePairedDelimiter\floor{\lfloor}{\rfloor} % floor function

\DeclarePairedDelimiter\paren{(}{)} % parenthesis

\newcommand{\df}{\displaystyle\frac} % displaystyle fraction
\newcommand{\qeq}{\overset{?}{=}} % questionable equality

\newcommand{\Mod}[1]{\;\mathrm{mod}\; #1} % modulo operator

% Sets
\DeclarePairedDelimiter\set{\{}{\}}
\newcommand{\unite}{\cup}
\newcommand{\inter}{\cap}

\newcommand{\reals}{\mathbb{R}}
\newcommand{\realspos}{\mathbb{R}^+} % real numbers: textbook is Z^+
\newcommand{\ints}{\mathbb{Z}}
\newcommand{\posints}{\mathbb{Z}^+}
\newcommand{\nats}{\mathbb{N}} % textbook is Z^+ and 0
\newcommand{\rats}{\mathbb{Q}}
\newcommand{\comps}{\mathbb{C}}

% Counting
\newcommand\perm[2][^n]{\prescript{#1\mkern-2.5mu}{}P_{#2}}
\newcommand\comb[2][^n]{\prescript{#1\mkern-0.5mu}{}C_{#2}}

\setlength\parindent{0pt}

% Sign Charts
\newdimen\tcolw \tcolw=2.5em % the column width
\edef\ecatcode{\catcode`&=\the\catcode`&\relax}\catcode`&=4
\def\sgchart#1#2{\vbox{\offinterlineskip\halign{\hfil##\quad&##\hfil\crcr\sgchartA#2,:,%
   \omit\sgchartR&\kern.2pt\sgchartS{.5\tcolw}\relax\sgchartE#1,\relax,%
   \sgchartS{.5\tcolw}\relax\cr
   \noalign{\kern2pt}&\def~{}\kern.5\tcolw\sgchartD#1,\relax,\cr}}}
\def\sgchartA#1:#2,{\cr\ifx,#1,\else $#1$&\sgchartB#2{}\expandafter\sgchartA\fi}
\def\sgchartB#1{\hbox to\tcolw{\hss$#1$\hss}\sgchartC}
\def\sgchartC#1{\ifx,#1,\else
   \strut\vrule\kern-.4pt\hbox to\tcolw{\hss$#1$\hss}\expandafter\sgchartC\fi}
\def\sgchartD#1#2,{\ifx\relax#1\else\hbox to\tcolw{\hss$#1#2$\hss}\expandafter\sgchartD\fi}
\def\sgchartE#1#2,{\ifx\relax#1\else
    \ifx~#1\sgchartS\tcolw\circ \else\sgchartS\tcolw\bullet\fi \expandafter\sgchartE\fi}
\def\sgchartR{\leaders\vrule height2.8pt depth-2.4pt\hfil}
\def\sgchartS#1#2{\hbox to#1{\kern-.2pt\sgchartR \ifx\relax#2\else
   \kern-.7pt$#2$\kern-.7pt\sgchartR\fi\kern-.2pt}}
\ecatcode
%++++++++++++++++++++++++++++++++++++++++
\title{
    \vspace{2in}
    \textmd{\textbf{\studyTitle}}
    \normalsize\vspace{0.1in}\\
    \vspace{0.1in}\large{\text{\class}} \\
    \vspace{0.1in}\text{\professor}\\
    \vspace{0.1in}\large\text{Final: \text{\final}}\\
    \vspace{3in}
}

\author{\name}
\date{\today}

\begin{document}
    \maketitle
    \thispagestyle{empty}
    \pagebreak
    \tableofcontents
    \pagebreak
    
    \section{Logic and Proofs}  
    \subsection{Propositional Logic}
    \stepcounter{subsubsection}
    \subsubsection{Propositions}
    \begin{definition}
        \textbf{Proposition:} A statement that is either true or false.
    \end{definition}    
    \begin{figure}[H]
        \centering
            \begin{tabular}{c|c}
                $p$ & $\neg p$ \\
                \hline
                T & F \\
                F & T
            \end{tabular}
        \caption{Truth table for \textbf{negation}}
    \end{figure}
    \begin{figure}[H]
        \centering
            \begin{tabular}{c|c|c|c|c}
                $p$ & $q$ & $p \land q$ & $p \lor q$ & $p \xor q$ \\
                \hline
                T & T & T & T & F \\
                T & F & F & T & T \\
                F & T & F & T & T \\
                F & F & F & F & F
            \end{tabular}
        \caption{Truth table for \textbf{bit operations}}
    \end{figure}
    \subsubsection{Conditional Statements}
    \begin{definition}
        \textbf{Conditional Statement:} A statement of the form $p \to q$. The conditional statement is called the \textit{hypothesis} (or \textit{antecedent} or \textit{premise}) and $q$ is called the \textit{conclusion} (or \textit{consequence}). 
    \end{definition}
    \begin{definition}
        \textbf{Converse:} The proposition $q \to p$ is the converse of the proposition $p \to q$. 
    \end{definition}
    \begin{figure}[H]
        \centering
            \begin{tabular}{c|c|c}
                $p$ & $q$ & $q \to p$ \\
                \hline
                T & T & T \\
                T & F & T \\
                F & T & F \\
                F & F & T 
            \end{tabular}   
        \caption{Truth Table for converse of implication of two propositions $p$ and $q$}
    \end{figure}
    \begin{definition}
        \textbf{Contrapositive:} The proposition $\neg q \to \neg p$ is the contrapositive of the proposition $p \to q$.  
    \end{definition}
    \begin{itemize}
        \item Same truth value as $p \to q$
    \end{itemize}
    \begin{figure}[H]
        \centering
            \begin{tabular}{c|c|c|c|c}
                $p$ & $q$ & $\neg{p}$ & $\neg{q}$ & $\neg q \to \neg p$ \\
                \hline
                T & T & F & F & T \\
                T & F & F & T & F \\
                F & T & T & F & T \\
                F & F & T & T & T
            \end{tabular}   
        \caption{Truth Table for contrapositive of implication of two propositions $p$ and $q$}
    \end{figure}
    \begin{definition}
        \textbf{Inverse:} The proposition $\neg p \to \neg q$ is the inverse of the proposition $p \to q$.    
    \end{definition}
    \begin{figure}[H]
        \centering
            \begin{tabular}{c|c|c|c|c}
                $p$ & $q$ & $\neg{p}$ & $\neg{q}$ & $\neg p \to \neg q$ \\
                \hline
                T & T & F & F & T \\
                T & F & F & T & T \\
                F & T & T & F & F \\
                F & F & T & T & T
            \end{tabular}    
        \caption{Truth Table for inverse of implication of two propositions $p$ and $q$}
    \end{figure}
    \stepcounter{subsubsection}
    \subsubsection{Precedence of Logical Operators}
    \begin{figure}[H]
        \centering
        \begin{tabular}{|c|c|}
            \hline
            Operator & Precedence \\
            \hline
            $\neg$ & 1 \\
            $\land$ & 2 \\
            $\lor$ & 3 \\
            $\to$ & 4 \\
            $\lra$ & 5 \\
            \hline        
        \end{tabular}
        \caption{Precedence of Logical Operators}
    \end{figure}
    \stepcounter{subsection}
    \subsection{Propositional Equivalences}
    \subsubsection{Introduction}
    \begin{definition}
        \textbf{Tautology:} A compound proposition that is always true.
    \end{definition}
    \begin{definition}
        \textbf{Contradiction:} A compound proposition that is always false.  
    \end{definition}
    \begin{definition}
        \textbf{Contingency:} A compound proposition that is neither a tautology nor a contradiction. 
    \end{definition}
    \begin{figure}[H]
        \centering
        \begin{tabular}{c|c|c|c}
            $p$ & $\neg q$ & $p \lor \neg q$ & $p \land \neg q$ \\
            \hline
            T & T & T & F \\
            T & F & T & F \\
        \end{tabular}
        \caption{Truth Table of an example of a Tautology and Contradiction}
    \end{figure}
    \subsubsection{Logical Equivalences}
    \begin{definition}
        Two propositions are \textbf{logically equivalent} if $p \lra q$ is a tautology.    
    \end{definition}
    The following are important logical equivalences:
    \begin{figure}[H]
        \centering
        \begin{tabular}{|c|}
            \hline
            De Morgan's Laws \\
            \hline
            \begin{tabular}{c}
                $\neg(p \land q) \lra \neg p \lor \neg q$ \\
                $\neg(p \lor q) \lra \neg p \land \neg q$ \\
            \end{tabular} \\
            \hline
        \end{tabular}
    \end{figure}
    \begin{figure}[H]
        \centering
        \begin{tabular}{|c|}
            \hline
            Conditional-Disjunction Equivalence \\
            \hline
            \begin{tabular}{c}
                $p \to q \lra \neg p \lor q$ \\
            \end{tabular} \\
            \hline
        \end{tabular}
    \end{figure}    

    Here are some other logical equivalences:
    \begin{figure}[H]
        \centering
        {\setlength{\tabcolsep}{2em}
        \begin{tabular}{|c|c|}
            \hline
            \multicolumn{2}{|c|}{} \\
            \multicolumn{2}{|c|}{Logical Equivalences} \\
            \multicolumn{2}{|c|}{} \\
            \hline
            Equivalence & Name \\
            \hline
            $p \land \true \equiv \true$ & Identity Laws \\
            $p \land \false \equiv \false$ &  \\
            \hline
            $p \lor \true \equiv \true$ & Domination Laws \\
            $p \lor \false \equiv \false$ &  \\
            \hline
            $p \lor p \equiv p$ & Idempotent Laws \\
            $p \land p \equiv p$ &  \\
            \hline
            $\neg(\neg p) \equiv p$ & Double Negation Law \\
            \hline
            $p \lor q \equiv q \lor p $ & Commutative Laws \\
            $p \land q \equiv q \land p$ &  \\
            \hline
            $(p \lor q) \lor r \equiv p \lor (q \lor r)$ & Associative Laws \\
            $(p \land q) \land r \equiv p \land (q \land r)$ &  \\
            \hline
            $p \lor (q \land r) \equiv (p \lor q) \land (p \lor r)$ & Distributive Laws \\
            $p \land (q \lor r) \equiv (p \land q) \lor (p \land r)$ &  \\
            \hline
            $\neg(p \land q) \equiv \neg p \lor \neg q$ & De Morgan's Laws \\
            $\neg(p \lor q) \equiv \neg p \land \neg q$ &  \\
            \hline
            $p \lor (p \land q) \equiv p$ & Absorption Laws \\
            $p \land (p \lor q) \equiv p$ &  \\
            \hline
            $p \lor \neg p \equiv \true$ & Negation Laws \\
            $p \land \neg p \equiv \false$ &  \\
            \hline
        \end{tabular}}
    \end{figure}
    
    \begin{figure}[H]
        \centering
        {\setlength{\tabcolsep}{2em}
        \begin{tabular}{|c|}
            \hline
            \\
            Logical Equivalences Involving Conditional Statements \\
            \\
            \hline
            $p \to q \equiv \neg p \lor q$ \\
            $p \to q \equiv \neg q \to \neg p$ \\
            $p \lor q \equiv \neg p \to q$ \\
            $p \land q \equiv \neg (\neg p \lor \neg q) \equiv \neg (p \to \neg q)$ \\
            $\neg(p \to q) \equiv p \land \neg q$ \\
            $(p \to q) \land (p \to r) \equiv p \to (q \land r) $ \\
            $(p \to r) \land (q \to r) \equiv (p \lor q) \to r$ \\
            $(p \to q) \lor (p \to r) \equiv p \to (q \lor r)$ \\
            $(p \to r) \lor (q \to r) \equiv (p \land q) \to r$ \\
            \hline
        \end{tabular}}
    \end{figure}
    \begin{figure}[H]
        \centering
        {\setlength{\tabcolsep}{2em}
        \begin{tabular}{|c|}
            \hline
            \\
            Logical Equivalences Involving Biconditonal Statements \\
            \\
            \hline
            $p \lra q \equiv (p \to q) \land (q \to p)$ \\
            $p \lra q \equiv \neg p \lra \neg q$ \\
            $p \lra q \equiv (p \land q) \lor (\neg p \land \neg q)$ \\
            $\neg(p \lra q) \equiv p \lra \neg q$ \\
            \hline
        \end{tabular}}
    \end{figure}
    \begin{equation*}
        \bigvee_{i=1}^n p_i = p_1 \lor p_2 \lor \cdots \lor p_n
    \end{equation*}
    \begin{equation*}
        \bigwedge_{i=1}^n p_i = p_1 \land p_2 \land \cdots \land p_n
    \end{equation*}
    By De Morgan's laws, it follows that:
    \stepcounter{subsubsection}
    \stepcounter{subsubsection}
    \begin{equation*}
        \neg \bigvee_{i=1}^n p_i = \bigwedge_{i=1}^n \neg p_i
    \end{equation*}
    \subsubsection{Satisfiability}
    \begin{definition}
        A compound proposition is \textbf{satisfiable} if there is an assignment of truth values to its variables that makes it true (When it is a tautology or a contigency). 
    \end{definition}
    \begin{definition}
        A compound proposition is \textbf{unsatisfiable} if there is no assignment of truth values to its variables that makes it true (When it is a contradiction). To prove this, we can prove that the negation is a tautology.
    \end{definition}
    \subsection{Predicates and Quantifiers}
    \stepcounter{subsubsection}
    \subsubsection{Predicates}
    \begin{definition}
        A \textbf{statement} contains 2 parts: a \textbf{subject} and a \textbf{predicate}.
        \begin{itemize}
            \item In the statement, $x$ is greater than 3, $x$ is the subject and greater than 3 is the predicate.
            \item The statement $P(x)$ is said to be the value of the \textbf{propositional function} $P$ at $x$.
        \end{itemize}
    \end{definition}
    \subsubsection{Quantifiers}
    \begin{definition}
        \textbf{Universal Quantifier}: $\forall x P(x)$. $P(x)$ for all values of $x$ in the domain.
        \begin{itemize}
            \item An element for which $P(x)$ is false is called a \textbf{counterexample}.
        \end{itemize}   
    \end{definition}
    \begin{definition}
        \textbf{Existential Quantifier}: $\exists x P(x)$. There exists an element $x$ in the domain such that $P(x)$ is true.    
    \end{definition}
    \begin{definition}
        \textbf{Uniqueness Quantifier}: $\exists! x P(x)$. There exists exactly one element $x$ in the domain such that $P(x)$ is true.    
    \end{definition}
    
    \textit{A way to think about determining the truth value of quantifiers is to think about looping. To determine if $\forall x P(x)$ is true, we loop through all the elements in the domain and check if $P(x)$ is true for all of them. To determine if $\exists x P(x)$ is true, we loop through all the elements in the domain and check if $P(x)$ is true for at least one of them.}
    \subsubsection{Quantifiers Over Finite Domains}
    When domain is finite, we can express statements using propositional logic:
    \begin{equation*}
        \forall x P(x) \equiv \bigwedge_{i=1}^{n} P(x)
    \end{equation*}
    \begin{equation*}
        \exists x P(x) \equiv \bigvee_{i=1}^{n} P(x)
    \end{equation*}
    \stepcounter{subsubsection}
    \subsubsection{Precedence of Quantifiers}
    $\forall x$ and $\exists x$ have higher precedence than all logical operators from propositional calculus. For instance, $\forall P(x) \lor Q(x) \equiv (\forall P(x)) \lor Q(x)$.
    \stepcounter{subsubsection}
    \subsubsection{Negating Quantified Expressions}
    \begin{figure}[H]
        \begin{align*}
            \neg \forall x P(x) &\equiv \exists x \neg P(x) \\
            \neg \exists x P(x) &\equiv \forall x \neg P(x)
        \end{align*}
        \caption{De Morgan's Laws for Quantifiers}
    \end{figure}
    \subsection{Nested Quantifiers}
    \begin{figure}[H]
        \centering
        \begin{tabular}{|c|p{12em}|p{12em}|}
            \hline
            Statement & When True & When False \\
            \hline
            $\forall x \forall y P(x,y)$ & $P(x,y)$ is true for all values of $x$ and $y$ & $\exists x \exists y \neg P(x,y)$ \\
            \hline
            $\forall x \exists y P(x,y)$ & For every $x$ there is a $y$ such that $P(x,y)$ is true & $\exists x \forall y \neg P(x,y)$ \\
            \hline
            $\exists x \forall y P(x,y)$ & There is an $x$ such that $P(x,y)$ is true for all values of $y$ & $\forall x \exists y \neg P(x,y)$ \\
            \hline
            $\exists x \exists y P(x,y)$ & There is an $x$ and a $y$ such that $P(x,y)$ is true & $\forall x \forall y \neg P(x,y)$ \\
            \hline
        \end{tabular}
        \caption{Quantifications of Two Variables}
    \end{figure}
    \subsection{Rules of Inference}
    \begin{figure}[H]
        \centering
        \begin{tabular}{|l|c|c|}
            \hline
            Rule of Inference & Tautology & Name \\
            \hline
            $\begin{array}{rl}
                & p \\
                & p \to q \\
                \cline{2-2}
                \therefore & q
            \end{array}$ & $(p \land (p \to q)) \to q$ & Modus Ponens \\
            \hline
            $\begin{array}{rl}
                & \neg q \\
                & p \to q \\
                \cline{2-2}
                \therefore & q 
            \end{array}$ & $(\neg q \land (p \to q)) \to \neg p$ & Modus Tollens \\
            \hline
            $\begin{array}{rl}
                & p \to q \\
                & q \to r \\
                \cline{2-2}
                \therefore & p \to r
            \end{array}$ & $((p \to q) \land (q \to r)) \to (p \to r)$ & Hypothetical Syllogism \\
            \hline
            $\begin{array}{rl}
                & p \lor q \\
                & \neg p \\
                \cline{2-2}
                \therefore & q
            \end{array}$ & $((p \lor q) \land \neg p) \to q$ & Disjunctive Syllogism \\
            \hline
            $\begin{array}{rl}
                & p \\
                \cline{2-2}
                \therefore & p \lor q
            \end{array}$ & $p \to (p \lor q)$ & Addition \\
            \hline
            $\begin{array}{rl}
                & p \land q \\
                \cline{2-2}
                \therefore & p
            \end{array}$ & $(p \land q) \to p$ & Simplification \\
            \hline
            $\begin{array}{rl}
                & p \\
                & q \\
                \cline{2-2}
                \therefore & p \land q
            \end{array}$ & $((p) \land (q)) \to (p \land q)$ & Conjunction \\
            \hline
            $\begin{array}{rl}
                & p \lor q \\
                & \neg p \lor r \\
                \cline{2-2}
                \therefore & q \lor r
            \end{array}$ & $((p \lor q) \land (\neg p \lor r)) \to (q \lor r)$ & Resolution \\
            \hline
        \end{tabular}
        \caption{Rules of Inference for Propositional Logic}
    \end{figure}
    Note: Resolution is saying, regardless of what $p$ is, $q$ or $r$ is true.
    \begin{figure}[H]
        \centering
        \begin{tabular}{|l|c|}
            \hline
            Rule of Inference & Name \\
            \hline
            $\begin{array}{rl}
                & \forall x P(x) \\
                \cline{2-2}
                \therefore & P(c)
            \end{array}$ & Universal Instantiation \\
            \hline
            $\begin{array}{rl}
                & P(c) \text{ for an arbitrary } c \\
                \cline{2-2}
                \therefore & \forall x P(x)
            \end{array}$ & Universal Generalization \\
            \hline
            $\begin{array}{rl}
                & \exists x P(x) \\
                \cline{2-2}
                \therefore & P(c) \text{ for some element } c
            \end{array}$ & Existential Instantiation \\
            \hline
            $\begin{array}{rl}
                & P(c) \text{ for some element } c \\
                \cline{2-2}
                \therefore & \exists x P(x)
            \end{array}$ & Existential Generalization \\
            \hline
        \end{tabular}
        \caption{Rules of Inference for Quantified Statements}
    \end{figure}
    \textbf{Universal Modus Ponens:} The usage of universal instantiation and modus ponens together.
    \begin{center}
        $\begin{array}{rl}
            & \forall x (P(x) \to Q(x)) \\
            & P(a) \text{, where } a \text{ is a particular element in the domain} \\
            \cline{2-2}
            \therefore & Q(a)
        \end{array}$
    \end{center}
    \subsection{Introduction to Proofs}
    Prove $\forall x(P(x) \to Q(x))$ by showing that $P(c) \to Q(c)$ is true, where $c$ is an arbitrary element of the domain, and then apply universal generalization.
    \stepcounter{subsubsection}
    \stepcounter{subsubsection}
    \stepcounter{subsubsection}
    \stepcounter{subsubsection}
    \subsubsection{Direct Proofs}
    Show $p \to q$ by first assuming $p$ is true and then showing that $q$ is true using the rules of inference.
    
    \begin{definition}
        The integer $n$ is \textbf{even} if $\exists k \in \ints \ | \ n = 2k$.
    \end{definition}
    \begin{definition}
        The integer $n$ is \textbf{odd} if $\exists k \in \ints \ | \ n = 2k + 1$.   
    \end{definition}
    \begin{definition}
        Two integers have the same \textbf{parity} if they are both even or both odd. They have \textbf{opposite parity} if one is even and the other is odd.     
    \end{definition}
    \subsubsection{Proof by Contraposition}
    \begin{definition}
        An \textbf{indirect proof} is a proof that does not start with the premise and ends with the conclusion. One type is \textbf{proof by contraposition}. This is a proof that shows $p \to q$ by showing $\neg q \to \neg p$.   
    \end{definition}
    \subsubsection{Proof by Contradiction}
    Another type of indirect proof is \textbf{proof by contradiction}. 
    We can prove $p$ is true by showing that $\neg p \to (r \land \neg r)$. Assume the negation of $p$ is true, and then show that $r$ and $\neg r$ are both true. This is a contradiction, so $p$ must be true.
    \\

    Important example of proof by contradiction: Prove that $\sqrt{2}$ is irrational.
    \begin{proof} By contradiction. Let $p$ be the proposition that $\sqrt{2}$ is irrational. Assum $\neg p$ is true, that is, assume $\sqrt{2}$ is rational. By definition of rational numbers, there exist integers $a$ and $b$ such that $\sqrt{2} = \frac{a}{b}$, where $\frac{a}{b}$ is in simplest terms. Then, we have:
    \begin{align*}
        \sqrt{2} &= \frac{a}{b} \\
               2 &= \frac{a^2}{b^2} \\
            2b^2 &= a^2 \\
        \intertext{Since $a^2$ is a multiple of 2, $a^2$ is even. Therefore, $a$ is even, and $a = 2k$ for some integer $k$.}
           2b^2 &= 4k^2 \\
            b^2 &= 2k^2 \\
        \intertext{Since $b^2$ is a multiple of 2, $b^2$ is even. Therefore, $b$ is even, and $b = 2l$ for some integer $l$.}
        \sqrt{2} &= \frac{2k}{2l}
    \end{align*}
    Since both $a$ and $b$ are even, we can divide both the numerator and denominator by 2. Because our assumption of $\neg p$ leads to the contradiction that 2 divides both $a$ and $b$ and 2 does not divide both $a$ and $b$, $\neg p$ is false. Therefore, $p$ is true.
    \end{proof}
    \subsection{Proof Methods and Strategy}
    \stepcounter{subsubsection}
    \subsubsection{Exhaustive Proof and Proof by Cases}
    \textbf{Exhaustive Proof}: Prove that $p \to q$ by showing:
    \begin{equation*}
        \bigvee_{i=1}^n p_i \to q
    \end{equation*}
    \textbf{Proof by Cases}: Prove that $p \to q$ by breaking $p$ into cases and showing that $q$ is true in each case.
    \section{Basic Structures}
    \subsection{Sets}
    \subsubsection{Introduction}
    \begin{definition}
        A \textbf{set} is an unordered collection of distinct objects called \textbf{elements} or \textbf{members} of the set. A set is said to \textbf{contain} its elements. We write $a \in A$ to denote that $a$ is an element of the set $A$. The notation $a \notin A$ denotes that $a$ is not an element of the set $A$.   
    \end{definition}

    Sets of types of numbers:
    \begin{itemize}
        \item Natural Numbers: $\nats = \{0, 1, 2, 3, \dots\} = \set*{\posints \unite 0}$
        \item Integers: $\ints = \{..., -2, -1, 0, 1, 2, \dots\}$
        \item Positive Integers: $\ints^+ = \{1, 2, 3, \dots\}$
        \item Rational Numbers: $\rats = \set*{\df{a}{b} \ | \ a, b \in \ints \text{ and } b \neq 0}$
        \item Real Numbers: $\reals$
        \item Positive Real Numbers: $\reals^+$
        \item Complex Numbers: $\comps$
    \end{itemize}

    \begin{definition}
        \textbf{Equality of Sets}:
        \begin{equation*}
            A=B \lra \forall x (x \in A \lra x \in B) \lra A \subseteq B \land B \subseteq A   
        \end{equation*}
    \end{definition}

    \begin{definition}
        \textbf{Empty Set}: $\emptyset = \set*{}$
    \end{definition}
    \stepcounter{subsubsection}
    \subsubsection{Subsets}
    \begin{definition}
        \textbf{Subset}:
        \begin{equation*}
            A \subseteq B \lra \forall x (x \in A \lra x \in B) \lra B \supseteq A
        \end{equation*}
        To show that $A \not\subseteq B$, show $\exists x (x \in A \land x \not\in B)$.
    \end{definition}

    \begin{theorem}
        For every set $S$, $\emptyset \subseteq S$ and $S \subseteq S$.
    \end{theorem}
    \subsubsection{Size of a Set}
    \begin{definition}
        Let $S$ be a set. If there are exactly $n$ distinct elements in $S$, where $n$ is a nonnegative integer, we say that $S$ is a \textbf{finite set} and that $n$ is the \textbf{cardinality} of $S$, denoted by $|S|$.
        \begin{itemize}
            \item Note: Theorem 2.1!
        \end{itemize}
    \end{definition}
    \subsubsection{Power Sets}
    \begin{definition}
        Let $S$ be a set. The \textbf{power set} of $S$, denoted by $\powset(S)$, is the set of all subsets of $S$.
    \end{definition}
    \begin{theorem} Cardinality of a power set
        \begin{equation*}
            |\powset(S)| = 2^{|S|}
        \end{equation*}
    \end{theorem}
    \subsubsection{Cartesian Products}
    \begin{definition}
        Let $A$ and $B$ be sets. The \textbf{Cartesian product} of $A$ and $B$, denoted by $A \times B$, is the set of all ordered pairs $(a,b)$ where $a \in A$ and $b \in B$. Hence:
        \begin{equation*}
            A \times B = \{(a,b) \ | \ a \in A \land b \in B\}    
        \end{equation*}
    \end{definition}
    \subsection{Set Operations}
    \subsubsection{Introduction}
    \begin{definition}
        Let $A$ and $B$ be sets. The \textbf{union} of the sets $A$ and $B$, denoted $A \unite B$, is the set that contains those elements that are in either $A$ or $B$ or both. Hence:
        \begin{equation*}
            A \unite B = \{x \ | \ x \in A \lor x \in B\}
        \end{equation*}
    \end{definition}
    \begin{definition}
        Let $A$ and $B$ be sets. The \textbf{intersection} of the sets $A$ and $B$, denoted $A \inter B$, is the set that contains those elements in both $A$ and $B$. Hence:
        \begin{equation*}
            A \inter B = \{x \ | \ x \in A \land x \in B\}
        \end{equation*}
    \end{definition}
    \begin{definition}
        Two sets are called \textbf{disjoint} if their intersection is the emptyset.
    \end{definition}
    \begin{definition}
        Let $A$ and $B$ be sets. The \textbf{difference} of the sets $A$ and $B$, denoted $A - B$, is the set that contains those elements in $A$ but not in $B$. It is also called the \textbf{complement of $B$ with respect to $A$}. Hence:
        \begin{equation*}
            A - B = \{x \ | \ x \in A \land x \not\in B\}
        \end{equation*}
    \end{definition}
    \begin{definition}
        Let $U$ be the universal set. The \textbf{complement} of a set $A$, denoted $\overline{A}$, is the set $U - A$.
    \end{definition}
    \subsubsection{Set Identities}
    \begin{figure}[H]
        \centering
        {\renewcommand{\arraystretch}{1.5}
        \begin{tabular}{|l|c|}
            \hline
            Identity & Name \\
            \hline
            $A \inter U = A$ & Identity Laws \\
            $A \unite \emptyset = A$ & \\
            \hline
            $A \unite U = U$ & Domination Laws \\
            $A \inter \emptyset = \emptyset$ & \\
            \hline
            $A \unite A = A$ & Idempotent Laws \\
            $A \inter A = A$ & \\
            \hline
            $\overline{(\overline{A})} = A$ & Complementation Law \\
            \hline
            $A \unite B = B \unite A$ & Commutative Laws \\
            $A \inter B = B \inter A$ & \\
            \hline
            $A \unite (B \unite C) = (A \unite B) \unite C$ & Associative Laws \\
            $A \inter (B \inter C) = (A \inter B) \inter C$ & \\
            \hline
            $A \unite (B \inter C) = (A \unite B) \inter (A \unite C)$ & Distributive Laws \\
            $A \inter (B \unite C) = (A \inter B) \unite (A \inter C)$ & \\
            \hline
            $\overline{A \inter B} = \overline{A} \unite \overline{B}$ & De Morgan's Laws \\
            $\overline{A \unite B} = \overline{A} \inter \overline{B}$ & \\
            \hline
            $A \unite (A \inter B) = A$ & Absorption Laws \\
            $A \inter (A \unite B) = A$ & \\
            \hline
            $A \unite \overline{A} = U$ & Complement Laws \\
            $A \inter \overline{A} = \emptyset$ & \\
            \hline
        \end{tabular}}
        \caption{Set Identities}
    \end{figure}
    There are 3 ways to prove that two sets are equal:
    \begin{enumerate}
        \item Showing that they are subsets of each other. (Definition 2.2)
        \item Membership tables.
        \item Set identities.
    \end{enumerate}
    A \textbf{membership table} considers each combination of the atomic sets (the original sets used to produce the sets on each side) that an element can belong to and verify that elements in the same combinations of sets belong to both the sets in the identity. Use a 1 to indicate that an element belongs to a set and a 0 to indicate that it does not. For example, consider the following identity:
    \begin{equation*}
        A \unite (A \inter B) = A
    \end{equation*}
    We can construct a membership table for this identity as follows:
    \begin{center}
        \begin{tabular}{|c|c|c|}
            \hline
            $A$ & $B$ & $A \unite (A \inter B)$ \\
            \hline
            1 & 1 & 1 \\
            1 & 0 & 1 \\
            0 & 1 & 0 \\
            0 & 0 & 0 \\
            \hline
        \end{tabular}
    \end{center}
    Since the columns are the same, we can conclude that the sets are equal.
    \subsubsection{Generalized Unions and Intersections}
    \begin{definition}
        The \textbf{union} of a collection of sets is the set that contains those elements that are members of at least one set in the collection. It is denoted by:
        \begin{equation*}
            A_1 \unite A_2 \unite \cdots A_n = \bigcup_{i=1}^{n} A_i
        \end{equation*}
    \end{definition}
    \begin{definition}
        The \textbf{intersection} of a collection of sets is the set that contains those elements that are members of all sets in the collection. It is denoted by:
        \begin{equation*}
            A_1 \inter A_2 \inter \cdots A_n = \bigcap_{i=1}^{n} A_i
        \end{equation*}
    \end{definition}
    \subsection{Functions}
    \subsection{Sequences and Summations}
    \subsection{Cardinality of Sets}
    \subsection{Matrices}
    \section{Number Theory}
    \subsection{Divisibility and Modular Arithmetic}
    \subsection{Integer Representations and Algorithms}
    \subsection{Primes and Greatest Common Divisors}
    \subsection{Solving Congruences}
    \subsection{Applications of Congruences}
    \section{Induction and Recursion}
    \subsection{Mathematical Induction}
    \subsection{Strong Induction and Well Ordering Principle}
    \section{Counting}
    \subsection{Basics of Counting}
    \subsection{Pigeonhole Principle}
    \subsection{Permutations and Combinations}
    \subsection{Binomial Coefficients and Identities}
    \subsection{Generalized Permutations and Combinations}
    \subsection{Generating Permutations and Combinations}
    \section{Probability}
    \subsection{Introduction to Discrete Probability}
    \subsection{Probability Theory}
    \subsection{Bayes' Theorem}
    \subsection{Expected Value and Variance} 
\end{document}