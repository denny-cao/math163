%++++++++++++++++++++++++++++++++++++++++
\documentclass[article, 11pt]{article}
\usepackage{float}
\usepackage{setspace}
\usepackage{tabu} % extra features for tabular environment
\usepackage{amsmath}  % improve math presentation
\usepackage{graphicx} % takes care of graphic including machinery
\usepackage[margin=1in]{geometry} % decreases margins
\usepackage{cite} % takes care of citations
\usepackage[final]{hyperref} % adds hyper links inside the generated pdf file
\usepackage{tikz}
\usepackage{caption} 
\usepackage{fancyhdr}
\usepackage{amssymb} % symbols like /therefore
\usepackage{amsthm} % proofs
\usepackage{enumerate} % lettered lists
\usepackage{mathtools} % macros
\usetikzlibrary{scopes}
\usepackage{xcolor} \pagecolor[rgb]{0.12549019607,0.1294117647,0.13725490196} \color[rgb]{0.82352941176,0.76862745098,0.62745098039} % dark theme
\hypersetup{
	colorlinks=false,       % false: boxed links; true: colored links
	linkcolor=blue,        % color of internal links
	citecolor=blue,        % color of links to bibliography
	filecolor=magenta,     % color of file links
	urlcolor=blue         
}
\usepackage{titling}
\renewcommand\maketitlehooka{\null\mbox{}\vfill}
\renewcommand\maketitlehookd{\vfill\null}
\usepackage{siunitx} % units
\usepackage{verbatim} 
\newcommand{\studyTitle}{Study Guide} 
\newcommand{\class}{MATH 163: Discrete Mathematics 1 Fall 2022}
\newcommand{\professor}{Dr. Petrescu}
\newcommand{\name}{Denny Cao}
\pagestyle{fancy}
\fancyhf{}% clears all header and footer fields
\fancyfoot[C]{--~\thepage~--}
\renewcommand*{\headrulewidth}{0.4pt}
\renewcommand*{\footrulewidth}{0pt}
\lhead{\name}
\chead{\class}
\rhead{\professor}


\fancypagestyle{plain}{%
  \fancyhf{}% clears all header and footer fields
  \fancyfoot[C]{--~\thepage~--}%
  \renewcommand*{\headrulewidth}{0pt}%
  \renewcommand*{\footrulewidth}{0pt}%
}

% Shortcuts
\newcommand{\xor}{\oplus} % exclusive or
\DeclarePairedDelimiter\ceil{\lceil}{\rceil} % ceil function
\DeclarePairedDelimiter\floor{\lfloor}{\rfloor} % floor function

\DeclarePairedDelimiter\paren{(}{)} % parenthesis

\newcommand{\df}{\displaystyle\frac} % displaystyle fraction
\newcommand{\qeq}{\overset{?}{=}} % questionable equality

\newcommand{\Mod}[1]{\;\mathrm{mod}\; #1} % modulo operator

% Sets
\DeclarePairedDelimiter\set{\{}{\}}
\newcommand{\unite}{\cup}
\newcommand{\inter}{\cap}

\newcommand{\reals}{\mathbb{R}} % real numbers: textbook is Z^+ and 0
\newcommand{\ints}{\mathbb{Z}}
\newcommand{\nats}{\mathbb{N}}
\newcommand{\rats}{\mathbb{Q}}


% Counting
\newcommand\perm[2][^n]{\prescript{#1\mkern-2.5mu}{}P_{#2}}
\newcommand\comb[2][^n]{\prescript{#1\mkern-0.5mu}{}C_{#2}}

\setlength\parindent{0pt}

% Sign Charts
\newdimen\tcolw \tcolw=2.5em % the column width
\edef\ecatcode{\catcode`&=\the\catcode`&\relax}\catcode`&=4
\def\sgchart#1#2{\vbox{\offinterlineskip\halign{\hfil##\quad&##\hfil\crcr\sgchartA#2,:,%
   \omit\sgchartR&\kern.2pt\sgchartS{.5\tcolw}\relax\sgchartE#1,\relax,%
   \sgchartS{.5\tcolw}\relax\cr
   \noalign{\kern2pt}&\def~{}\kern.5\tcolw\sgchartD#1,\relax,\cr}}}
\def\sgchartA#1:#2,{\cr\ifx,#1,\else $#1$&\sgchartB#2{}\expandafter\sgchartA\fi}
\def\sgchartB#1{\hbox to\tcolw{\hss$#1$\hss}\sgchartC}
\def\sgchartC#1{\ifx,#1,\else
   \strut\vrule\kern-.4pt\hbox to\tcolw{\hss$#1$\hss}\expandafter\sgchartC\fi}
\def\sgchartD#1#2,{\ifx\relax#1\else\hbox to\tcolw{\hss$#1#2$\hss}\expandafter\sgchartD\fi}
\def\sgchartE#1#2,{\ifx\relax#1\else
    \ifx~#1\sgchartS\tcolw\circ \else\sgchartS\tcolw\bullet\fi \expandafter\sgchartE\fi}
\def\sgchartR{\leaders\vrule height2.8pt depth-2.4pt\hfil}
\def\sgchartS#1#2{\hbox to#1{\kern-.2pt\sgchartR \ifx\relax#2\else
   \kern-.7pt$#2$\kern-.7pt\sgchartR\fi\kern-.2pt}}
\ecatcode
%++++++++++++++++++++++++++++++++++++++++
\title{
    \vspace{2in}
    \textmd{\textbf{\studyTitle}}
    \normalsize\vspace{0.1in}\\
    \vspace{0.1in}\large{\text{\class: \professor}}
    \vspace{3in}
}

\author{\name}
\date{\today}

\begin{document}
    \maketitle
    \thispagestyle{empty}
    \pagebreak
    \tableofcontents
    \pagebreak
    
    \section{Logic and Proofs}  
    \subsection{Propositional Logic}
    \subsubsection{Truth Tables of Logical Operators}
    \textbf{Proposition:} A statement that is either true or false.
    \begin{figure}[H]
        \centering
            \begin{tabular}{c|c}
                $p$ & $\neg p$ \\
                \hline
                T & F \\
                F & T
            \end{tabular}
        \caption{Truth table for \textbf{negation}}
    \end{figure}
    \begin{figure}[H]
        \centering
            \begin{tabular}{c|c|c|c|c}
                $p$ & $q$ & $p \land q$ & $p \lor q$ & $p \xor q$ \\
                \hline
                T & T & T & T & F \\
                T & F & F & T & T \\
                F & T & F & T & T \\
                F & F & F & F & F
            \end{tabular}
        \caption{Truth table for \textbf{bit operations}}
    \end{figure}
    \subsubsection{Converse, Contrapositive, and Inverse}
    \textbf{Converse:} The proposition $q \Rightarrow p$ is the converse of the proposition $p \Rightarrow q$. 
    \begin{figure}[H]
        \centering
            \begin{tabular}{c|c|c}
                $p$ & $q$ & $q \Rightarrow p$ \\
                \hline
                T & T & T \\
                T & F & T \\
                F & T & F \\
                F & F & T 
            \end{tabular}   
        \caption{Truth Table for converse of implication of two propositions $p$ and $q$}
    \end{figure}
    \textbf{Contrapositive:} The proposition $\neg q \Rightarrow \neg p$ is the contrapositive of the proposition $p \Rightarrow q$.
    \begin{itemize}
        \item Same truth value as $p \Rightarrow q$
    \end{itemize}
    \begin{figure}[H]
        \centering
            \begin{tabular}{c|c|c|c|c}
                $p$ & $q$ & $\neg{p}$ & $\neg{q}$ & $\neg q \Rightarrow \neg p$ \\
                \hline
                T & T & F & F & T \\
                T & F & F & T & F \\
                F & T & T & F & T \\
                F & F & T & T & T
            \end{tabular}   
        \caption{Truth Table for contrapositive of implication of two propositions $p$ and $q$}
    \end{figure}
    \textbf{Inverse:} The proposition $\neg p \Rightarrow \neg q$ is the inverse of the proposition $p \Rightarrow q$.
    \begin{figure}[H]
        \centering
            \begin{tabular}{c|c|c|c|c}
                $p$ & $q$ & $\neg{p}$ & $\neg{q}$ & $\neg p \Rightarrow \neg q$ \\
                \hline
                T & T & F & F & T \\
                T & F & F & T & T \\
                F & T & T & F & F \\
                F & F & T & T & T
            \end{tabular}    
        \caption{Truth Table for inverse of implication of two propositions $p$ and $q$}
    \end{figure}
    \subsubsection{Precedence of Logical Operators}
    \begin{figure}[H]
        \centering
        \begin{tabular}{|c|c|}
            \hline
            Operator & Precedence \\
            \hline
            $\neg$ & 1 \\
            $\land$ & 2 \\
            $\lor$ & 3 \\
            $\Rightarrow$ & 4 \\
            $\Leftrightarrow$ & 5 \\
            \hline        
        \end{tabular}
        \caption{Precedence of Logical Operators}
    \end{figure}
    \subsection{Applications of Propositional Logic}
    \begin{enumerate}
        \item Translating English into Propositional Logic and vice versa
        \item Logic Puzzles 
    \end{enumerate}
    Intuitive. 
    \subsection{Propositional Equivalences}
    \subsubsection{Introduction}
    \textbf{Tautology:} A compound proposition that is always true.

    \textbf{Contradiction:} A compound proposition that is always false.

    \textbf{Contingency:} A compound proposition that is neither a tautology nor a contradiction.

    \begin{figure}[H]
        \centering
        \begin{tabular}{c|c|c|c}
            $p$ & $\neg q$ & $p \lor \neg q$ & $p \land \neg q$ \\
            \hline
            T & T & T & F \\
            T & F & T & F \\
        \end{tabular}
        \caption{Truth Table of an example of a Tautology and Contradiction}
    \end{figure}
    \subsubsection{Logical Equivalences}
    Two propositions are \textbf{logically equivalent} if $p \Leftrightarrow q$ is a tautology. 
    The following are important logical equivalences:
    \begin{figure}[H]
        \centering
        \begin{tabular}{|c|}
            \hline
            De Morgan's Laws \\
            \hline
            \begin{tabular}{c}
                $\neg(p \land q) \Leftrightarrow \neg p \lor \neg q$ \\
                $\neg(p \lor q) \Leftrightarrow \neg p \land \neg q$ \\
            \end{tabular} \\
            \hline
        \end{tabular}
        \caption{De Morgan's Laws}
    \end{figure}
    \textbf{Conditional-disjunction equivalence:} $p \lor q \Leftrightarrow \neg p \Rightarrow q$
    \subsection{Predicates and Quantifiers}
    \subsection{Nested Quantifiers}
    \subsection{Rules of Inference}
    \subsection{Introduction to Proofs}
    \subsection{Proof Methods and Strategy}
\end{document}