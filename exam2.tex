%++++++++++++++++++++++++++++++++++++++++
\documentclass[article,12pt]{article}
\usepackage{float}
\usepackage{setspace}
\usepackage{tabu} % extra features for tabular environment
\usepackage{amsmath}  % improve math presentation
\usepackage{graphicx} % takes care of graphic including machinery
\usepackage[margin=1in]{geometry} % decreases margins
\usepackage{cite} % takes care of citations
\usepackage[final]{hyperref} % adds hyper links inside the generated pdf file
\usepackage{tikz}
\usepackage{caption} 
\usepackage{fancyhdr}
\usepackage{amssymb} % symbols like /therefore
\usepackage{amsthm} % proofs
\usepackage{enumerate} % lettered lists
\usepackage{mathtools} % macros
\usetikzlibrary{scopes}
\usepackage{xcolor} \pagecolor[rgb]{0.12549019607,0.1294117647,0.13725490196} \color[rgb]{0.82352941176,0.76862745098,0.62745098039}
\hypersetup{
	colorlinks=true,       % false: boxed links; true: colored links
	linkcolor=blue,        % color of internal links
	citecolor=blue,        % color of links to bibliography
	filecolor=magenta,     % color of file links
	urlcolor=blue         
}

\newcounter{question}
\setcounter{question}{0}
\def\questionnum{{\Large\bfseries{Question \arabic{question} }}\\[1em]}

\newcommand{\question}{
    \stepcounter{question}
    \questionnum
}

\newcommand{\solution}{
    {{\large\bfseries{Solution}}}
}

\pagestyle{fancy}
\fancyhf{}% clears all header and footer fields
\fancyfoot[C]{--~\thepage~--}
\renewcommand*{\headrulewidth}{0pt}
\renewcommand*{\footrulewidth}{0pt}

\fancypagestyle{plain}{%
  \fancyhf{}% clears all header and footer fields
  \fancyfoot[C]{--~\thepage~--}%
  \renewcommand*{\headrulewidth}{0pt}%
  \renewcommand*{\footrulewidth}{0pt}%
}

% Shortcuts
\DeclarePairedDelimiter\ceil{\lceil}{\rceil}
\DeclarePairedDelimiter\floor{\lfloor}{\rfloor}
\DeclarePairedDelimiter\set{\{}{\}}
\DeclarePairedDelimiter\paren{(}{)}

\newcommand{\df}{\displaystyle\frac} % displaystyle fraction
\newcommand{\qeq}{\overset{?}{=}} % questionable equality

\newcommand{\comp}{\circ}

% Sets
\newcommand{\unite}{\cup}
\newcommand{\inter}{\cap}

\newcommand{\stcomp}[1]{\overline{#1}}  

\newcommand{\reals}{\mathbb{R}} % real numbers: textbook is Z^+ and 0
\newcommand{\ints}{\mathbb{Z}}
\newcommand{\nats}{\mathbb{N}}
\newcommand{\rats}{\mathbb{Q}}

\setlength\parindent{0pt}

%++++++++++++++++++++++++++++++++++++++++

\title{Exam 2 Corrections}
\author{Denny Cao}
\date{%
    %CCP MATH163 \\%
    %Dr. Petrescu \\[2ex]%
    \today
}
\begin{document}
\maketitle

% Question 1
\question
If $\stcomp{A} \inter B = B \ \land \ \stcomp{B}\inter A = A$, what can you say about $A$ and $B$? Prove your answer. ($\stcomp{S}$ denotes the complement of $S$.) \\
\\
\solution
\begin{proof}
    Assume $x \in A$. By definition of stcomplement, $x \notin \stcomp{A}$. Since $x \notin \stcomp{A}$, $x \notin \stcomp{A} \inter B$. Since $\stcomp{A} \inter B = B$, $x \notin B$. $x$ is arbitrary, therefore we can generalize to $A \not\subseteq{B}$.
\end{proof}
\begin{proof}
    Assume $y \in B$. By definition of ststcomplement, $y \notin \stcomp{B}$. Since $y \notin \stcomp{B}$, $y \notin \stcomp{B} \inter A$. Since $\stcomp{B} \inter A = A$, $y \notin A$. $y$ is arbitrary, therefore we can generalize to $B \not\subseteq A$.
\end{proof}
\begin{proof}
    Since $A \not\subseteq B \ \land \ B \not\subseteq A$, $A \inter B = \emptyset \\
    \\
    \stcomp{A} \inter B = B \ \land \ \stcomp{B} \inter A = A \implies A \inter B = \emptyset$
\end{proof}
\newpage
% Question 2
\question
Suppose that $g : A \rightarrow A$ and $f: A \rightarrow A$ where $A=\set*{a,b,c,d}$, $g=\set*{(a,a),(b,c),(c,a),(d,c)}$ and $f=\set*{(a,d),(b,a),(c,b),(d,a)}$. Find $g \comp (f \comp g)$, that is, find the composition of $g(x)$ with $f(x)$ composed with $g(x)$. \\
\\
\solution
\begin{align*}
    &g(A) = \set*{(a,a),(b,c),(c,a),(d,c)} \\
    &f(g(A)) = \set*{(a,d),(b,b),(c,d),(d,b)} \\
    &g(f(g(A))) = \set*{(a,c),(b,c),(c,c),(d,c)}
\end{align*}
\newpage
% Quesiton 3
\question 
Verify that $a_n = \pi$ \textbf{is a possible solution} to the recurrence relation $a_n = (2n-1)a_{n-1}-(2n-2)a_{n-2}$. What are the conditions for this to happen? \\
\\
\solution
\newpage
% Question 4
\question
Evaluate $\sum_{n=1}^{3}\sum_{k=1}^{n}(n^k)$. Show all your steps. \\
\\
\solution
\begin{align*}
    \sum_{n=1}^{3}\sum_{k=1}^{n}(n^k) &= \sum_{k=1}^{1}{1^k}+\sum_{k=1}^{2}{2^k}+ \sum_{k=1}^{3}{3^k} \\
                                      &= 1 + 2 + 4 + 3 + 9 + 27 \\
                                      &= 46
\end{align*}
\newpage
% Question 5
\question
Find a $f: \nats \rightarrow \ints$ that is surjection that is not injection. Prove that it satisfies the conditions held. \\
\\
\solution \\
Let $S_1 = \set*{x: 2x, x \in \nats}$, $S_2 = \set*{x: 2x-1, x \in \nats}$
\[ f(n)=
\begin{cases}
    \frac{n}{2} & n \in S_1 \\
    -(\frac{n}{2}+\frac{1}{2}) & n \in S_2
\end{cases}
\]
\textbf{Theorem: $f(n)$ is surjective.}
\begin{proof}
    We can separate the domain, $\nats$ into two sets of odd and even naturals, $S_1$ and $S_2$ respectively. \\
    Let $n=2|y|$. Since $2|y|$ is always an even, it will satisfy the first case, $n \in S_1$. 
\end{proof}
\end{document} 
