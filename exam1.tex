%++++++++++++++++++++++++++++++++++++++++
\documentclass[letterpaper,12pt]{article}
\usepackage{float}
\usepackage{setspace}
\usepackage{tabu} % extra features for tabular environment
\usepackage{amsmath}  % improve math presentation
\usepackage{graphicx} % takes care of graphic including machinery
\usepackage[margin=1in,letterpaper]{geometry} % decreases margins
\usepackage{cite} % takes care of citations
\usepackage[final]{hyperref} % adds hyper links inside the generated pdf file
\usepackage{tikz}
\usepackage{caption} 
\usepackage{fancyhdr}
\usepackage{amssymb} % symbols like /therefore
\usepackage{amsthm} % proofs

\pagestyle{fancy}
\fancyhf{}% clears all header and footer fields
\fancyfoot[C]{--~\thepage~--}
\renewcommand*{\headrulewidth}{0pt}
\renewcommand*{\footrulewidth}{0pt}

\fancypagestyle{plain}{%
  \fancyhf{}% clears all header and footer fields
  \fancyfoot[C]{--~\thepage~--}%
  \renewcommand*{\headrulewidth}{0pt}%
  \renewcommand*{\footrulewidth}{0pt}%
}

\captionsetup[table]{skip=15px}

\usetikzlibrary{scopes}

\hypersetup{
	colorlinks=true,       % false: boxed links; true: colored links
	linkcolor=blue,        % color of internal links
	citecolor=blue,        % color of links to bibliography
	filecolor=magenta,     % color of file links
	urlcolor=blue         
}
%++++++++++++++++++++++++++++++++++++++++
\begin{document}
\onehalfspacing

\setlength\parindent{0pt}
\title{Exam 1 Corrections}
\author{Denny Cao}
\date{October 9, 2022}
\maketitle
% Question 1
\textbf{Q1. The domain is the set of all real numbers. Determine the truth value of the following expression: $\forall x \exists y (x < 0 \lor y^2 = x)$} \\
When $x < 0$, it satisfies the first condition, $x<0$. When $x \geq 0$, $y=\sqrt{x}$ satisfies the second condition, $y^2=x$. \\ 
$\forall x \exists y|x < 0 \rightarrow x < 0 \land \forall x \exists y|x \geq 0 \rightarrow y = \sqrt{x}$ \\
The expression is true. \\
\\
% Question 2
\textbf{Q2. Prove that $\neg[p \lor (s \land (\neg p \rightarrow q))] \equiv \neg p \land (\neg s \lor \neg q)$ using a truth table.}

\begin{center}
    \begin{tabular}{|c|c|c|c|c|c|c|c|}
        \hline
        $p$&$q$&$s$&$\neg p$&$\neg p \rightarrow q$&$s \wedge \left(\neg p \rightarrow q\right)$&$p \vee \left(s \wedge \left(\neg p \rightarrow q\right)\right)$&$\neg [p \vee \left(s \wedge \left(\neg p \rightarrow q\right)\right)]$\\ \hline
        T&T&T&F&T&T&T&F\\\hline
        T&T&F&F&T&F&T&F\\\hline
        T&F&T&F&T&T&T&F\\\hline
        T&F&F&F&T&F&T&F\\\hline
        F&T&T&T&T&T&T&F\\\hline
        F&T&F&T&T&F&F&T\\\hline
        F&F&T&T&F&F&F&T\\\hline
        F&F&F&T&F&F&F&T\\\hline
    \end{tabular}
\end{center}
\begin{center}
    \begin{tabular}{|c|c|c|c|c|c|c|c|}
        \hline
        $p$&$q$&$s$&$\neg s$&$\neg q$&$\neg q \vee \neg s$&$\neg p$&$\neg p \wedge \left(\neg q \vee \neg s\right)$\\ \hline
        T&T&T&F&F&F&F&F\\\hline
        T&T&F&T&F&T&F&F\\\hline
        T&F&T&F&T&T&F&F\\\hline
        T&F&F&T&T&T&F&F\\\hline
        F&T&T&F&F&F&T&F\\\hline
        F&T&F&T&F&T&T&T\\\hline
        F&F&T&F&T&T&T&T\\\hline
        F&F&F&T&T&T&T&T\\\hline
    \end{tabular}
\end{center}
Since the truth values of $\neg [p \lor (s \land (\neg p \rightarrow q))]$ are the same as $\neg p \land (\neg s \lor \neg q)$, the two statements are equivalent. \\
\\
% Question 3
\textbf{Q3. Write the statement:\textit{Everybody loves Raymond} and its negation symbolically using quantifiers. Clearly state what variable represents what statement.} \\
Let $L(x)$ be the statement that $x$ loves Raymond, and the domain consists of everyone. \\
Original: $$\forall x (L(x))$$
Negation: $$\neg(\forall x (L(x)) \equiv \exists x | \neg L(x)$$ 
\\
% Question 4
\textbf{Q4. Consider the theorem \textit{The second power of integers end only in the digits} $\{0,1,4,5,6,9\}$} \\
\textbf{a. Write the statement of this theorem \underline{symbolically}}, as a conditional, as if you were to use a direct proof. \\
Let $P(x)$ be the statement that $x$ is the second power of an integer, and the domain consists of integers.
$$\forall x \in \mathbb{Z} (P(x) \rightarrow x \bmod{10} \in \{0,1,4,5,6,9\})$$
\textbf{b. Write the statement \underline{symbolically}, as a conditional, as if you were to prove it by contraposition.} 
$$\forall x \in \mathbb{Z} (x \bmod{10} \notin \{0,1,4,5,6,9\} \rightarrow \neg P(x))$$
% Question 5
\textbf{Q5. Suppose that the universe for $x$ and $y$ is $\{a,b,c,d\}$. Assume that $P(x,y)$ is a predicate that is true in the following cases: $P(a,d), P(b,a), P(b,b), P(b,c), P(c,a), P(c,b), \\ P(c,d), P(d,a), P(d,b), P(d,d)$ and false otherwise. \\
Determine and justify whether each of the following is true or false:} \\
\textbf{a. $\forall x (P(x,y) \rightarrow y \neq x$} \\
False. $x=b$. For $P(b,b)$, $y=x$. \\
\textbf{b. $\forall x \exists y|((P(x,y) \lor P(y,x))$} \\
True. $\forall x \exists y (P(x,y)). \ x=a, y=d \ P(a,d). \ x=b, y = a \ P(b,a). \ x=c, y=a \ P(c,a). \\ x=d, y=a \ P(d,a).$ \\
\textbf{c. $\forall y \exists x|(y \neq x \land P(x,y))$} \\
True. $y=a, x=b \ P(b,a). \ y=b, x=c \ P(c,b). \ y=c, x=b \ P(c,b). \ y=d, x=a \ P(a,d).$ \\
\\
% Question 6
\textbf{Q6. Negate this statement symbolically: \textit{No work, no credit without using it is not true that}... \\
Clearly indicate which variable is which statement.} \\
Let $w, c$ be the propositions that "there is no work", "there is no credit", respectively. \\
Original: $$w \rightarrow c$$
Negation: 
\begin{align*}
    \neg ( w \rightarrow c) &\equiv \neg (\neg w \lor c) \\
                            &\equiv w \land \neg c
\end{align*}
% Question 7
\textbf{Q7. The domain of discourse are the students in a class. Define the predicates: \\ $S(x)$ and $A(x)$ as \textit{$x$ received as an A on the test,} respectively. \\ 
Write symbolically: \textit{Someone who did not study for the test received an A on the test.}} \\
$$\exists x | (\neg S(x) \land A(x))$$ \\
% Question 8
\textbf{Q8. Determine whether the following argument is valid. \\
\\
If I get a job then I will buy a new car or a new house. \\
\underline{I won't buy a new house.} \\
$\therefore$ I will not get a job.} \\
\\
Let $j,c,h$ be the propositions that "I get a job", "I will buy a new car", "I will by a new house", respectively.
\begin{align}
    &j \rightarrow (c \lor h)               &&\text{Premise} \\
    &\neg h                                 &&\text{Premise} \\
    &\neg (c \lor h) \rightarrow \neg j     &&\text{Law of Contraposition from (1)} \\
    &\neg c \land \neg h \rightarrow \neg j &&\text{De Morgan's Laws from (3)}
\end{align}
False. $\neg h$ is not sufficient to prove $\neg j$; we must know the truth value of $c$ as well. \\
% Question 9
\newpage
\textbf{Q9. Prove: \textit{The average of two odd integers is an integer.}}
\begin{proof} Direct. Let $a = 2c+1, b=2k+1, c,k \in \mathbb{Z}$
\begin{align*}
\frac{a+b}{2} &= \frac{2c+1 + 2k+1}{2} \\
              &= \frac{2c + 2k + 2}{2} \\
              &= c + k + 1 
\end{align*}
The sum of integers is an integer.
\end{proof}
% Question 10
\textbf{Q10. Prove by contraposition: \textit{For every integer $n, n^2 - 2n + 7$ is even, then $n$ is odd}.}
\begin{proof} Contraposition. For every integer $n$, if $n$ is not odd, then $n^2 - 2n + 7$ is not even. \\
If $n$ is not odd, $n$ is even and can be rewritten as $2c, c \in \mathbb{Z}$
\begin{align*}
    (2c)^2 - 2(2c) + 7 &= 4c^2 - 4c + 7 \\
                       &= 4c^2 - 4c + 6 + 1 \\
                       &= 2(2c^2 - 2c + 3) + 1 \\
                       &\text{Let } 2c^2 - 2c + 3 = k \\
                       &= 2k + 1 
\end{align*}
Since $n^2 - 2n + 7$ can be rewritten as $2k + 1$, it is not even; it is odd.
\end{proof}
\end{document}
 